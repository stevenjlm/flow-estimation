%%%%%%%%%%%%%%%%%%%%%%%%%%%%%%%%%%%%%%%%%
% Beamer Presentation
% LaTeX Template
% Version 1.0 (10/11/12)
%
% This template has been downloaded from:
% http://www.LaTeXTemplates.com
%
% License:
% CC BY-NC-SA 3.0 (http://creativecommons.org/licenses/by-nc-sa/3.0/)
%
%%%%%%%%%%%%%%%%%%%%%%%%%%%%%%%%%%%%%%%%%

%----------------------------------------------------------------------------------------
%                                                                     PACKAGES AND THEMES
%----------------------------------------------------------------------------------------
\pdfminorversion=4
\documentclass{beamer}

\mode<presentation> {

%                                            The Beamer class comes with a number of
%                                            default slide themes which change the colors
%                                            and layouts of slides. Below this is a list
%                                            of all the themes, uncomment each in turn to
%                                            see what they look like.

%\usetheme{default}
%\usetheme{AnnArbor}
%\usetheme{Antibes}
%\usetheme{Bergen}
%\usetheme{Berkeley}
%\usetheme{Berlin}
%\usetheme{Boadilla}
%\usetheme{CambridgeUS}
%\usetheme{Copenhagen}
%\usetheme{Darmstadt}
%\usetheme{Dresden}
%\usetheme{Frankfurt}
%\usetheme{Goettingen}
%\usetheme{Hannover}
%\usetheme{Ilmenau}
%\usetheme{JuanLesPins}
%\usetheme{Luebeck}
\usetheme{Madrid}
%\usetheme{Malmoe}
%\usetheme{Marburg}
%\usetheme{Montpellier}
%\usetheme{PaloAlto}
%\usetheme{Pittsburgh}
%\usetheme{Rochester}
%\usetheme{Singapore}
%\usetheme{Szeged}
%\usetheme{Warsaw}

%                                            As well as themes, the Beamer class has a number
%                                            of color themes for any slide theme. Uncomment
%                                            each of these in turn to see how it changes the
%                                            colors of your current slide theme.

%\usecolortheme{albatross}
%\usecolortheme{beaver}
%\usecolortheme{beetle}
%\usecolortheme{crane}
%\usecolortheme{dolphin}
%\usecolortheme{dove}
%\usecolortheme{fly}
%\usecolortheme{lily}
%\usecolortheme{orchid}
%\usecolortheme{rose}
%\usecolortheme{seagull}
%\usecolortheme{seahorse}
%\usecolortheme{whale}
%\usecolortheme{wolverine}

%\setbeamertemplate{footline} %              To remove the footer line in all
%                                            slides uncomment this line
%\setbeamertemplate{footline}[page number] % To replace 
%                                            the footer line in all slides with a simple
%                                            slide count uncomment this line

%\setbeamertemplate{navigation symbols}{} %  To remove the
%                                            navigation symbols from the bottom of all 
%                                            slides uncomment this line
}

%----------------------------------------------------------------------------------------
%                                                                           Custom Colors
%----------------------------------------------------------------------------------------
\AtBeginSection{\frame{\sectionpage}}

\definecolor{UC_blue}{RGB}{18,149,216}

\setbeamercolor{structure}{fg=UC_blue}


\usepackage{graphicx} %                      Allows including images
\usepackage{booktabs} %                      Allows the use of \toprule, \midrule
%                                            and \bottomrule in tables
\usepackage{bm}

%----------------------------------------------------------------------------------------
%	                                                                       Title Page
%----------------------------------------------------------------------------------------

%                                            The short title appears at the bottom of
%                                            every slide, the title is only on the
%                                            title page
\title[Network Process Modeling]{Parameter Estimation for Network Processes}

\author{Steven Munn} %                       Your name

%                                            Your institution as it will appear on the
%                                            bottom of every slide, may be shorthand to
%                                            save space
\institute[UCSB]
{
University of California, Santa Barbara \\ % Your institution for the title page
\medskip
\textit{sjmunn@umail.ucsb.edu} %             Your email address
}
\date{\today} %                              Date, can be changed to a custom date

\begin{document}

\begin{frame}
\titlepage %                                 Print the title page as the first slide
\end{frame}

\begin{frame}
\frametitle{Overview} %                      Table of contents slide, comment 
%                                            this block out to remove it
\tableofcontents %                           Throughout your presentation, if
%                                            you choose to use \section{} and
%                                            \subsection{} commands, these will
%                                            automatically be printed on this slide
%                                            as an overview of your presentation
\end{frame}

%----------------------------------------------------------------------------------------
%	                                                              Presentation Slides
%----------------------------------------------------------------------------------------

%------------------------------------------------
\section{Graph Attributes}

\begin{frame}
\frametitle{Attributed Graphs}
%                                            We are interested in representing problems
%                                            as graphs with attributes
\begin{figure}
\includegraphics[width=0.8\linewidth]{figs/attributed_graph}
\end{figure}
\end{frame}

\begin{frame}
\frametitle{Traffic Example}
\begin{figure}
\includegraphics[width=0.8\linewidth]{figs/trafficEg}
\end{figure}
\end{frame}

\begin{frame}
\frametitle{Website Visits Example}
\begin{figure}
\includegraphics[width=0.8\linewidth]{figs/websitesEg}
\end{figure}
\end{frame}
%------------------------------------------------
\section{Reaction Networks}

\begin{frame}
\frametitle{Reaction Attributes}
\begin{figure}
\includegraphics[width=0.8\linewidth]{figs/reaction_graph}
\end{figure}
\end{frame}

\begin{frame}
\frametitle{Graph Weights}
\begin{columns}
\begin{column}{0.3\textwidth}
\[
\mathbf{W}=\left[\begin{array}{ccc}
w_{00} & w_{01} & w_{02}\\
w_{10} & w_{11} & w_{12}\\
w_{20} & w_{21} & w_{22}
\end{array}\right]
\]
\end{column}
\begin{column}{0.7\textwidth}  %%<--- here
    \begin{center}
     \includegraphics[width=0.5\textwidth]{figs/reacWeights}
     \end{center}
\end{column}
\end{columns}
\end{frame}

\begin{frame}
\frametitle{Reaction ODE}
\[
dM^{T}=\left[\begin{array}{cccc}
1 & 1 & 1 & ...\end{array}\right]\times\left[\begin{array}{cccc}
w_{00} & w_{01} & w_{02} & ...\\
w_{10} & w_{11} & w_{12} & ...\\
w_{20} & w_{21} & w_{22} & ...\\
... &  &  & ...
\end{array}\right]\otimes\left[\begin{array}{cccc}
1 & M_{0} & M_{0} & ...\\
M_{1} & 1 & M_{1} & ...\\
M_{2} & M_{2} & 1 & ...\\
... &  &  & ...
\end{array}\right]
\]
\end{frame}

\begin{frame}
\frametitle{Example: Homework One}
\begin{eqnarray*}
dM^{T} & = & \left[\begin{array}{ccc}
1 & 1 & 1\end{array}\right]\times\left[\begin{array}{ccc}
1 & 0 & \theta_{1}\\
0 & 0 & \theta_{1}\\
0 & \theta_{2} & 0
\end{array}\right]\otimes\left[\begin{array}{ccc}
1 & M_{0} & M_{0}\\
M_{1} & 1 & M_{1}\\
M_{2} & M_{2} & 1
\end{array}\right]\\
 & = & \left[\begin{array}{ccc}
1 & 1 & 1\end{array}\right]\times\left[\begin{array}{ccc}
1 & 0 & \theta_{1}M_{0}\\
0 & 0 & \theta_{1}M_{1}\\
0 & \theta_{2}M_{2} & 0
\end{array}\right]\\
 & = & \left[\begin{array}{ccc}
1 & \theta_{2}M_{2} & \theta_{1}\left(M_{0}+M_{1}\right)\end{array}\right]
\end{eqnarray*}
\end{frame}

%------------------------------------------------
\section{Gibbs Sampling}

\begin{frame}
\frametitle{Multiple Parameters Estimation}
Start with parameters $\bm{\theta}^{\left( t\right)}$ (e.g. sampled from a uniform distribution)
Update each component one after the other as follows:
\begin{itemize}
\item Sample $\theta^{\left( t + 1 \right)}_{1}$ from $p \left( \theta_{1} | \text{data}, \theta_{2}, \theta_{3}, ... \right)$
\item Sample $\theta^{\left( t+1 \right)}_{2}$ from $p \left( \theta_{2} | \text{data}, \theta_{1}, \theta_{3}, ... \right)$
\item ...
\end{itemize}
\end{frame}

\begin{frame}
\frametitle{Gibbs Sampling for Reaction Networks}

\begin{block}{Step 1}
Define an upper and lower bound ( $\bm{W}_{up}$ and $\bm{W}_{low}$ ) on the graph weights.
\end{block}

\begin{block}{Step 2}
Initialize the probability for each parameter as a uniform distribution between the upper and lower bounds,
\[
p\left(w_{ij}\right)=u\left(\left( \bm{W}_{up} \right)_{ij}, \left( \bm{W}_{low} \right)_{ij} \right)
\]
\end{block}

\end{frame}

\begin{frame}
\frametitle{Gibbs Sampling for Reaction Networks}

We now have an initial guess for the weights matrix $\bm{W}$. Next, we will update each component one at a time.

\end{frame}

\begin{frame}
\frametitle{Gibbs Update for first component}
We need to compute,
\[
p\left(w_{00}|\text{data},w_{01},w_{02},w_{10},w_{11},...\right)
\]
to sample a new value for $w_{00}$. Currently, we do this using the same Bayesian inference method from homework one.
\end{frame}

\begin{frame}
\frametitle{An example with two parameters}
True parameters,
\[
\mathbf{W}=\left[\begin{array}{ccc}
1 & 0 & 1\\
0 & 0 & 1\\
0 & -1.45 & 0
\end{array}\right]
\]

\begin{columns}
\begin{column}{0.5\textwidth}

\[
\mathbf{W_{lower}}=\left[\begin{array}{ccc}
1 & 0 & 0\\
0 & 0 & 0\\
0 & -1.45 & 0
\end{array}\right]
\]

\end{column}
\begin{column}{0.5\textwidth}  %%<--- here

\[
\mathbf{W_{upper}}=\left[\begin{array}{ccc}
1 & 0 & 2\\
0 & 0 & 2\\
0 & -1.45 & 0
\end{array}\right]
\]

\end{column}
\end{columns}
\end{frame}

\begin{frame}
\frametitle{$w_{02}$ Estimate at step 1}
\begin{figure}
\includegraphics[width=0.8\linewidth]{figs/1_WeDistr}
\end{figure}
\end{frame}

\begin{frame}
\frametitle{$w_{02}$ Estimate at step 8}
\begin{figure}
\includegraphics[width=0.8\linewidth]{figs/8_WeDistr}
\end{figure}
\end{frame}

\begin{frame}
\frametitle{$w_{02}$ Estimate at step 9}
\begin{figure}
\includegraphics[width=0.8\linewidth]{figs/9_WeDistr}
\end{figure}
\end{frame}

\begin{frame}
\frametitle{$w_{02}$ Estimate at step 10}
\begin{figure}
\includegraphics[width=0.8\linewidth]{figs/10_WeDistr}
\end{figure}
\end{frame}

\begin{frame}
\frametitle{$w_{02}$ Estimate at step 11}
\begin{figure}
\includegraphics[width=0.8\linewidth]{figs/11_WeDistr}
\end{figure}
\end{frame}

\begin{frame}
\frametitle{$w_{02}$ Estimate at step 12}
\begin{figure}
\includegraphics[width=0.8\linewidth]{figs/12_WeDistr}
\end{figure}
\end{frame}

\begin{frame}
\frametitle{Gibbs Sampling Problems}
\begin{block}{Probability density in parameter space is too focused}
{Most of the parameter space has near-zero probability. The space over plausible parameters is very small. This is espcially true for runaway reactions (where the rates are all positive).}
\end{block}
\end{frame}

\end{document}
